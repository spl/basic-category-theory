% Basic Category Theory
% Tom Leinster <Tom.Leinster@ed.ac.uk>
% 
% Copyright (c) Tom Leinster 2014-2016
% 
% Chapter 1: Categories, functors and natural transformations
% 

\chapter{Categories, functors and natural transformations}
\label{ch:cfnt}

A category is a system of related objects.  The objects do not live in
isolation: there is some notion of map between objects, binding them
together.

Typical examples of what `object' might mean are `group' and `topological
space', and typical examples of what `map' might mean are `homomorphism'
and `continuous map', respectively.  We will see many examples, and we will
also learn that some categories have a very different flavour from the two
just mentioned.  In fact, the `maps' of category theory need not be
anything like maps in the sense that you are most likely to be familiar
with.

Categories are \emph{themselves} mathematical objects, and with that in
mind, it is unsurprising that there is a good notion of `map between
categories'.  Such maps are called functors.  More surprising, perhaps, is
the existence of a third level: we can talk about maps between
\emph{functors}, which are called natural transformations.  These, then,
are maps between maps between categories.

In fact, it was the desire to formalize the notion of natural
transformation that led to the birth of category theory.  By the early
1940s, researchers in algebraic topology had started to use the phrase
`natural transformation', but only in an informal way.  Two mathematicians,
Samuel Eilenberg%
%
\index{Eilenberg, Samuel}
%
and Saunders Mac Lane,%
%
\index{Mac~Lane, Saunders}
%
saw that a precise definition was needed.  But before they could define
natural transformation, they had to define functor; and before they could
define functor, they had to define category.  And so the subject was born.

Nowadays, the uses of category theory have spread far beyond algebraic
topology.  Its tentacles extend into most parts of pure mathematics.  They
also reach some parts of applied mathematics; perhaps most notably,
category theory has become a standard tool in certain parts of computer%
%
\index{computer science}
%
science.  Applied%
%
\index{applied mathematics}
% 
mathematics is more than just applied differential equations!



\section{Categories}
\label{sec:cats}


\begin{defn}
A \demph{category}%
%
\index{category}
%
$\cat{A}$ consists of:
% 
\begin{itemize}
\item 
a collection $\ob(\cat{A})$%
%
\ntn{ob}
%
of \demph{objects};%
%
\index{object}
%

\item 
for each $A, B \in \ob(\cat{A})$, a collection $\cat{A}(A, B)$%
%
\ntn{hom-set-default}
%
of \demph{maps}%
%
\index{map}
%
or \demph{arrows}%
%
\index{arrow}
%
or \demph{morphisms}%
%
\index{morphism}
% 
from $A$ to $B$;

\item 
for each $A, B, C \in \ob(\cat{A})$, a function
\[
\begin{array}{ccc}
\cat{A}(B, C) \times \cat{A}(A, B) &
\to	&
\cat{A}(A, C)	\\
(g, f)	&
\mapsto	&
g \of f,%
%
\ntn{of}
%
\end{array}
\]
called \demph{composition};%
%
\index{composition}
%

\item 
for each $A \in \ob(\cat{A})$, an element $1_A$%
%
\ntn{id-map}
%
of $\cat{A}(A, A)$, called the \demph{identity}%
%
\index{identity}
%
on $A$,
\end{itemize}
% 
satisfying the following axioms:
% 
\begin{itemize}
\item 
\demph{associativity}:%
%
\index{associativity}
%
for each $f \in \cat{A}(A, B)$, $g \in \cat{A}(B, C)$ and $h \in \cat{A}(C,
D)$, we have $(h \of g) \of f = h \of (g \of f)$;

\item 
\demph{identity%
%
\index{identity}
%
laws}: for each $f \in \cat{A}(A, B)$, we have $f \of 1_A = f = 1_B \of f$.
\end{itemize}
\end{defn}

\begin{remarks}  
\label{rmks:defn-cat}
\begin{enumerate}[(b)]
\item   
\label{item:defn-cat-notn}
We often write:
%
\begin{displaytext}
\begin{tabular}{rcl}
$A \in \cat{A}$			&to mean	&$A \in \ob(\cat{A})$;	\\
$f\from A\to B$ or $A \toby{f} B$&to mean 	&$f \in \cat{A}(A,B)$;%
%
\ntn{arrow}
%
\\
$gf$				&to mean	&$g \of f$.%
%
\ntn{juxt}
%
\end{tabular}
\end{displaytext}
% 
People also write $\cat{A}(A, B)$ as $\Hom_{\cat{A}}(A, B)$%
%
\ntn{Hom}
%
or $\Hom (A, B)$.  The notation `$\Hom$' stands for homomorphism, from one
of the earliest examples of a category.

\item	
\label{rmk:defn-cat:loosely}
The definition of category is set up so that in general, from each string
\[
A_0 \toby{f_1} 
A_1 \toby{f_2}
\ \cdots \ 
\toby{f_n} A_n
\]
of maps in $\cat{A}$, it is possible to construct exactly one%
%
\index{uniqueness!constructions@of constructions}
%
map
\[
A_0 \to A_n
\]
(namely, $f_n f_{n - 1} \cdots f_1$).  If we are given extra information
then we may be able to construct other maps $A_0 \to A_n$; for instance, if
we happen to know that $A_{n - 1} = A_n$, then $f_{n - 1} f_{n - 2} \cdots
f_1$ is another such map.  But we are speaking here of the \emph{general}
situation, in the absence of extra information.

For example, a string like this with $n = 4$ gives rise to maps
\[
\xymatrix@=8em{
A_0
\ar@<1ex>[r]^{((f_4 f_3)f_2)f_1}
\ar@<-1ex>[r]_{(f_4(1_{A_3} f_3))((f_2 f_1)1_{A_0})} &
A_4,
}
\]
but the axioms imply that they are equal.  It is safe to omit the brackets
and write both as $f_4 f_3 f_2 f_1$.

Here it is intended that $n \geq 0$.  In the case $n = 0$, the statement is
that for each object $A_0$ of a category, it is possible to construct
exactly one map $A_0 \to A_0$ (namely, the identity $1_{A_0}$).  An
identity map can be thought of as a zero-fold%
%
\index{identity!zero-fold composite@as zero-fold composite}
%
composite, in much the same way that the number $1$ can be thought of as
the product of zero numbers.

\item 
We often speak of \demph{commutative%
%
\index{diagram!commutative}
%
diagrams}.  For instance, given objects and maps
\[
\xymatrix{
A \ar[rr]^f \ar[d]_h    &               &B \ar[d]^g     \\
C \ar[r]_i              &D \ar[r]_j     &E
}
\]
in a category, we say that the diagram \demph{commutes}%
%
\index{commutes}
%
if $gf = jih$.  Generally, a diagram is said to commute if whenever there
are two paths from an object $X$ to an object $Y$, the map from $X$ to $Y$
obtained by composing along one path is equal to the map obtained by
composing along the other.

\item 
The slightly vague word `collection'%
%
\index{collection}
%
means \emph{roughly} the same as `set', although if you know about such
things, it is better to interpret it as meaning `class'.%
%
\index{class}
%
We come back to this in Chapter~\ref{ch:sets}.

\item 
If $f \in \cat{A}(A, B)$, we call $A$ the \demph{domain}%
%
\index{domain}
%
and $B$ the \demph{codomain}%
%
\index{codomain}
%
of $f$.  Every map in every category has a definite domain and a definite
codomain.  (If you believe it makes sense to form the intersection of an
arbitrary pair of abstract sets, you should add to the definition of
category the condition that $\cat{A}(A, B) \cap \cat{A}(A', B') =
\emptyset$ unless $A = A'$ and $B = B'$.)
\end{enumerate}
\end{remarks}

\begin{examples}[Categories \:of \:mathematical \:structures]        
\label{egs:cats-of}
% 
\begin{enumerate}[(b)]
\item 
There is a category $\Set$%
%
\ntn{Set}
%
described as follows.  Its objects are sets.%
%
\index{set!category of sets}
%
Given sets $A$ and $B$, a map from $A$ to $B$ in the category $\Set$ is
exactly what is ordinarily called a map (or mapping, or function) from $A$
to $B$.  Composition in the category is ordinary composition of functions,
and the identity maps are again what you would expect.

In situations such as this, we often do not bother to specify the
composition and identities.  We write `the category of sets and functions',
leaving the reader to guess the rest.  In fact, we usually go further and
call it just `the category of sets'.

\item 
There is a category $\Grp$%
%
\ntn{Grp}
%
of groups,%
%
\index{group!category of groups}
%
whose objects are groups and whose maps are group homomorphisms.

\item 
Similarly, there is a category $\Ring$%
%
\ntn{Ring}
%
of rings%
%
\index{ring!category of rings}
%
and ring homomorphisms.

\item 
For each field $k$, there is a category $\Vect_k$%
%
\ntn{Vect}
%
of vector%
%
\index{vector space!category of vector spaces}
%
spaces over $k$ and linear maps between them.

\item 
There is a category $\Tp$%
%
\ntn{Top}
%
of topological%
%
\index{topological space!category of topological spaces}
%
spaces and continuous maps.
\end{enumerate}
\end{examples}

This chapter is mostly about the interaction \emph{between} categories,
rather than what goes on \emph{inside} them.  We will, however, need the
following definition.

\begin{defn}    
\label{defn:isomorphism}
A map $f\from A \to B$ in a category $\cat{A}$ is an \demph{isomorphism}%
%
\index{isomorphism}
%
if there exists a map $g\from B \to A$ in $\cat{A}$ such that $gf = 1_A$
and $fg = 1_B$.
\end{defn}

In the situation of Definition~\ref{defn:isomorphism}, we call $g$ the
\demph{inverse}%
%
\index{inverse}
%
of $f$ and write $g = f^{-1}$.%
%
\ntn{inverse}
%
(The word `the' is justified by Exercise~\ref{ex:unique-inverse}.)  If
there exists an isomorphism from $A$ to $B$, we say that $A$ and $B$ are
\demph{isomorphic} and write $A \iso B$.%
%
\ntn{iso}
%

\begin{example}
\label{eg:iso-Set}
The isomorphisms in $\Set$%
%
\index{set!category of sets!isomorphisms in}
%
are exactly the bijections.  This\linebreak statement is not quite a logical
triviality.  It amounts to the assertion that a function has a two-sided
inverse if and only if it is injective and surjective.
\end{example}

\begin{example}
The isomorphisms in $\Grp$%
%
\index{group!category of groups!isomorphisms in}
%
are exactly the isomorphisms of groups.  Again, this is not quite trivial,
at least if you were taught that the definition of group isomorphism is
`bijective homomorphism'.  In order to show that this is equivalent to
being an isomorphism in $\Grp$, you have to prove that the inverse of a
bijective homomorphism is also a homomorphism.

Similarly, the isomorphisms in $\Ring$%
%
\index{ring!category of rings!isomorphisms in}
%
are exactly the isomorphisms of rings.
\end{example}

\begin{example}
The isomorphisms in $\Tp$%
%
\index{topological space!category of topological spaces!isomorphisms in}
%
are exactly the homeomorphisms.  Note that, in contrast to the situation in
$\Grp$ and $\Ring$, a bijective map in $\Tp$ is not necessarily an
isomorphism.  A classic example is the map
\[
\begin{array}{ccc}
[0, 1)  &\to            &\{z \in \complexes \such \left|z\right| = 1\}     \\
t       &\mapsto        &e^{2\pi i t},
\end{array}
\]
which is a continuous bijection but not a homeomorphism.
\end{example}

The examples of categories mentioned so far are important, but could give a
false impression.  In each of them, the objects of the category are sets
with structure (such as a group structure, a topology, or, in the case
of $\Set$, no structure at all).  The maps are the functions preserving the
structure, in the appropriate sense.  And in each of them, there is a
clear sense of what the elements of a given object are.

However, not all categories are like this.  In general, the objects of a
category are not `sets equipped with extra stuff'.  Thus, in a general
category, it does not make sense to talk about the `elements' of an object.
(At least, it does not make sense in an immediately obvious way; we return
to this in Definition~\ref{defn:gen-elt}.)  Similarly, in a general
category, the maps need not be mappings or functions in the usual sense.
So:

\begin{slogan}
The objects of a category need not be remotely like sets.%
%
\index{object!need not resemble set}
%
\end{slogan}
% 
\begin{slogan}
The maps in a category need not be remotely like functions.
%
\index{map!need not resemble function}
%
\end{slogan}
% 
The next few examples illustrate these points.  They also show that,
contrary to the impression that might have been given so far, categories
need not be enormous.  Some categories are small, manageable structures in
their own right, as we now see.

\begin{examples}[Categories \,as \,mathematical \,structures]        
\label{egs:cats-as}
\begin{enumerate}[(b)]
\item   
\label{eg:cats-as:graphs}
A category can be specified%
%
\index{category!drawing of}
%
by saying directly what its objects, maps, composition and identities are.
For example, there is a category $\emptyset$%
%
\ntn{empty-cat}
%
with no objects or maps at all.  There is a category $\One$%
%
\ntn{terminal-cat}
%
 with one object
and only the identity map.  It can be drawn like this:
\[
\bullet
\]
(Since every object is required to have an identity map on it, we usually
do not bother to draw the identities.)  There is another category that can
be drawn as
\[
\bullet \to \bullet 
\qquad
\text{or}
\qquad
A \toby{f} B,
\]
with two objects and one non-identity map, from the first object to the
second.  (Composition is defined in the only possible way.)  To reiterate
the points made above, it is not obvious what an `element' of $A$ or $B$
would be, or how one could regard $f$ as a `function' of any sort.

It is easy to make up more complicated examples.  For instance, here are three
more categories:
\[
\begin{array}{c}
\xymatrix{
\bullet \ar@<.5ex>[r] \ar@<-.5ex>[r] &\bullet
}
\end{array}
\qquad
\begin{array}{c}
\xymatrix{
        &B \ar[dr]^g    &       \\
A \ar[ur]^f \ar[rr]_{gf} & &C    
}
\end{array}
\qquad
\begin{array}{c}
\xymatrix{
        &\bullet \ar[dl]_{kj} \ar[r]^f \ar[dr]|{hj=gf} \ar[d]_{j} &
\bullet \ar[d]^g        \\
\bullet &\bullet \ar[l]^k \ar[r]_{h}   &\bullet
}
\end{array}
\]

\item   
\label{eg:cats-as:discrete} 
Some categories contain no maps at all apart from identities (which, as
categories, they are obliged to have).  These are called \demph{discrete}%
%
\index{category!discrete}
%
categories.  A discrete category amounts to just a class of objects.  More
poetically, a category is a collection of objects related to one another to
a greater or lesser degree; a discrete category is the extreme case in
which each object is totally isolated from its companions.

\item   
\label{eg:cats-as:groups}
A group is essentially the same thing as a category that has only one%
%
\index{category!one-object|(}%
\index{group!one-object category@as one-object category}
%
object and in which all the maps are isomorphisms.

To understand this, first consider a category $\cat{A}$ with just one
object.  It is not important what letter or symbol we use to denote the
object; let us call it $A$.  Then $\cat{A}$ consists of a set (or class)
$\cat{A}(A, A)$, an associative composition function
\[
\of\from  \cat{A}(A, A) \times \cat{A}(A, A) \to \cat{A}(A, A),
\]
and a two-sided unit $1_A \in \cat{A}(A, A)$.  This would make $\cat{A}(A,
A)$ into a group, except that we have not mentioned inverses.  However, to
say that every map in $\cat{A}$ is an isomorphism is exactly to say that
every element of $\cat{A}(A, A)$ has an inverse with respect to $\of$.

If we write $G$ for the group $\cat{A}(A, A)$, then the situation is this:
% 
\begin{displaytext}
\begin{tabular}{l@{\hspace{2em}}l}
\emph{category $\cat{A}$ with single object $A$}       &
\emph{corresponding group $G$}        \\[1ex]
maps in $\cat{A}$       &elements of $G$                \\
$\of$ in $\cat{A}$      &$\cdot$ in $G$                 \\
$1_A$                   &$1 \in G$      \\
\end{tabular}
\end{displaytext}
% 
The category $\cat{A}$ looks something like this:
\[
\SelectTips{cm}{}
\xymatrix{A \ar@(r,u)@<-.5ex>[]_{} \ar@(u,l) \ar@(ld,rd)[]_{}}
\]
The arrows represent different maps $A \to A$, that is, different elements of
the group $G$. 

What the object of $\cat{A}$ is called makes no difference.  It matters
exactly as much as whether we choose $x$ or $y$ or $t$ to denote some
variable in an algebra problem, which is to say, not at all.  Later we will
define `equivalence' of categories, which will enable us to make a precise
statement: the category of groups is equivalent to the category of (small)
one-object categories in which every map is an isomorphism
(Example~\ref{eg:mon-one-obj-eqv}).

The first time one meets the idea that a group is a kind of category, it is
tempting to dismiss it as a coincidence or a trick.  But it is not; there
is real content.

To see this, suppose that your education had been shuffled and that you
already knew about categories before being taught about groups.  In your
first group theory class, the lecturer declares that a group is supposed to
be the system of all symmetries of an object.  A symmetry of an object $X$,
she says, is a way of mapping $X$ to itself in a reversible or invertible
manner.  At this point, you realize that she is talking about a very
special type of category.  In general, a category is a system consisting of
\emph{all} the mappings (not usually just the invertible ones) between
\emph{many} objects (not usually just one).  So a group is just a category
with the special properties that all the maps are invertible and there is
only one object.

\item   
\label{eg:cats-as:monoids}
The inverses played no essential part in the previous example, suggesting that
it is worth thinking about `groups without inverses'.  These are called
monoids.  

Formally, a \demph{monoid}%
%
\index{monoid}
%
is a set equipped with an associative binary operation and a two-sided unit
element.  Groups describe the reversible transformations, or symmetries,
that can be applied to an object; monoids describe the
not-necessarily-reversible transformations.  For instance, given any set
$X$, there is a group consisting of all bijections $X \to X$, and there is
a monoid consisting of all functions $X \to X$.  In both cases, the binary
operation is composition and the unit is the identity function on $X$.
Another example of a monoid is the set $\nat = \{0, 1, 2, \ldots\}$%
%
\ntn{nat}
%
of natural%
%
\index{natural numbers}
%
numbers, with $+$ as the operation and $0$ as the unit.  Alternatively, we
could take the set $\nat$ with $\cdot$ as the operation and $1$ as the
unit.

A category with one%
%
\index{monoid!one-object category@as one-object category}
%
object is essentially the same thing as a monoid, by the same argument as
for groups.  This is stated formally in Example~\ref{eg:mon-one-obj-eqv}.%
%
\index{category!one-object|)}
%

\item   
\label{eg:cats-as:orders}
A \demph{preorder}%
%
\index{preorder}
%
is a reflexive transitive binary relation.  A \demph{preordered set} $(S,
\mathord{\leq})$%
%
\ntn{leq}
%
is a set $S$ together with a preorder $\leq$ on it.  Examples: $S = \reals$
and $\leq$ has its usual meaning; $S$ is the set of subsets of $\{1,
\ldots, 10\}$ and $\leq$ is $\sub$ (inclusion); $S = \integers$ and $a \leq
b$ means that $a$ divides $b$.

A preordered set can be regarded as a category $\cat{A}$ in which, for each
$A, B \in \cat{A}$, there is at most one map from $A$ to $B$.  To see this,
consider a category $\cat{A}$ with this property.  It is not important what
letter we use to denote the unique map from an object $A$ to an object $B$;
all we need to record is which pairs $(A, B)$ of objects have the property
that a map $A \to B$ does exist.  Let us write $A \leq B$ to mean that
there exists a map $A \to B$.

Since $\cat{A}$ is a category, and categories have composition, if $A \leq
B \leq C$ then $A \leq C$.  Since categories also have identities, $A \leq
A$ for all $A$.  The associativity and identity axioms are automatic.  So,
$\cat{A}$ amounts to a collection of objects equipped with a transitive
reflexive binary relation, that is, a preorder.  One can think of the
unique map $A \to B$ as the statement or assertion that $A \leq B$.

An \demph{order}%
%
\index{ordered set}
%
on a set is a preorder $\leq$ with the property that if $A \leq B$ and $B
\leq A$ then $A = B$.  (Equivalently, if $A \iso B$ in the corresponding
category then $A = B$.)  Ordered sets are also called \demph{partially
ordered sets}%
%
\index{partially ordered set} 
%
or \demph{posets}.%
%
\index{poset}
%
An example of a preorder that is not%
%
\index{ordered set!preordered set@vs.\ preordered set}
%
an order is the divisibility relation $\divides$ on $\integers$: for there
we have $2 \divides {-2}$ and $-2 \divides 2$ but $2 \neq -2$.
\end{enumerate}
\end{examples}

Here are two ways of constructing new categories from old. 

{\sloppy
\begin{constn} 
\label{constn:op-cat}
Every category $\cat{A}$ has an \demph{opposite}%
%
\index{category!opposite}
%
or \demph{dual}%
%
\index{duality}
%
category $\cat{A}^\op$,%
%
\ntn{op}
%
defined by reversing the arrows.  Formally, $\ob(\cat{A}^\op) =
\ob(\cat{A})$ and $\cat{A}^\op(B, A) = \cat{A}(A, B)$ for all objects $A$
and $B$.  Identities in $\cat{A}^\op$ are the same as in $\cat{A}$.
Composition in $\cat{A}^\op$ is the same as in $\cat{A}$, but with the
arguments reversed.  To spell this out: if $A \toby{f} B \toby{g} C$ are
maps in $\cat{A}^\op$ then $A \otby{f} B \otby{g} C$ are maps in $\cat{A}$;
these give rise to a map $A \otby{f \of g} C$ in $\cat{A}$, and the
composite of the original pair of maps is the corresponding map $A \to C$
in $\cat{A}^\op$.

So, arrows $A \to B$ in $\cat{A}$ correspond to arrows $B \to A$ in
$\cat{A}^\op$.  According to the definition above, if $f\from A \to B$ is
an arrow in $\cat{A}$ then the corresponding arrow $B \to A$ in
$\cat{A}^\op$ is also called $f$.  Some people prefer to give it a
different name, such as $f^\op$.
\end{constn}
}

\begin{remark}  
\label{rmk:principle-duality}
The \demph{principle of duality}%
%
\index{duality!principle of}
%
is fundamental to category theory.  Informally, it states that every
categorical definition, theorem and proof has a \demph{dual}, obtained by
reversing all the arrows.  Invoking the principle of duality can save work:
given any theorem, reversing the arrows throughout its statement and proof
produces a dual theorem.  Numerous examples of duality appear throughout
this book.
\end{remark}

\begin{constn}  
\label{constn:prod-cat}
Given categories $\cat{A}$ and $\cat{B}$, there is a \demph{product%
%
\index{category!product of categories}
%
category} $\cat{A} \times \cat{B}$,%
%
\ntn{prod-cat}
%
 in which
% 
\begin{align*}
\ob(\cat{A} \times \cat{B})     &
=      
\ob(\cat{A}) \times \ob(\cat{B}),\\
(\cat{A} \times \cat{B})((A, B), (A', B'))      &
=      
\cat{A}(A, A') \times \cat{B}(B, B').
\end{align*}
% 
Put another way, an object of the product category $\cat{A} \times \cat{B}$
is a pair $(A, B)$ where $A \in \cat{A}$ and $B \in \cat{B}$.  A map $(A,
B) \to (A', B')$ in $\cat{A} \times \cat{B}$ is a pair $(f, g)$ where
$f\from A \to A'$ in $\cat{A}$ and $g\from B \to B'$ in $\cat{B}$.  For the
definitions of composition and identities in $\cat{A} \times \cat{B}$, see
Exercise~\ref{ex:prod-cat}.
\end{constn}


\exs


\begin{question}
Find three examples of categories not mentioned above.
\end{question}


\begin{question}        
\label{ex:unique-inverse}
Show that a map in a category can have at most one inverse.  That is, given
a map $f\from A \to B$, show that there is at most one map $g\from B \to A$
such that $gf = 1_A$ and $fg = 1_B$.  
\end{question}


\begin{question}        
\label{ex:prod-cat}
Let $\cat{A}$ and $\cat{B}$ be categories.
Construction~\ref{constn:prod-cat} defined the product category $\cat{A}
\times \cat{B}$, except that the definitions of composition and identities
in $\cat{A} \times \cat{B}$ were not given.  There is only one sensible way
to define them; write it down.
\end{question}


\begin{question}
There is a category $\Toph$%
%
\ntn{Toph}
%
whose objects are topological spaces and whose
maps $X \to Y$ are homotopy%
%
\index{homotopy}
%
classes of continuous maps from $X$ to $Y$.  What do you need to know about
homotopy in order to prove that $\Toph$ is a category?  What does it mean,
in purely topological terms, for two objects of $\Toph$ to be isomorphic?
\end{question}



\section{Functors}
\label{sec:ftrs}


One of the lessons of category theory is that whenever we meet a new type
of mathematical object, we should always ask whether there is a sensible
notion of `map' between such objects.  We can ask this about categories
themselves.  The answer is yes, and a map between categories is called a
functor.

\begin{defn}
Let $\cat{A}$ and $\cat{B}$ be categories.  A \demph{functor}%
%
\index{functor}
%
$F\from \cat{A} \to \cat{B}$ consists of:
% 
\begin{itemize}
\item 
a function
\[
\ob(\cat{A}) \to \ob(\cat{B}),
\]
written as $A \mapsto F(A)$;

\item 
for each $A, A' \in \cat{A}$, a function
\[
\cat{A}(A, A') \to \cat{B}(F(A), F(A')),
\]
written as $f \mapsto F(f)$,
\end{itemize}
% 
satisfying the following axioms:
% 
\begin{itemize}
\item 
$F(f' \of f) = F(f') \of F(f)$ whenever $A \toby{f} A' \toby{f'} A''$ in
$\cat{A}$;

\item 
$F(1_A) = 1_{F(A)}$ whenever $A \in \cat{A}$.
\end{itemize}
\end{defn}

\begin{remarks} 
\label{rmks:defn-ftr}
\begin{enumerate}[(b)]       
\item 
\label{rmk:defn-ftr:loosely}
The definition of functor is set up so that from each string
\[
A_0 \toby{f_1} \ \cdots\ \toby{f_n} A_n
\]
of maps in $\cat{A}$ (with $n \geq 0$), it is possible to construct exactly
one%
%
\index{uniqueness!constructions@of constructions}
%
map
\[
F(A_0) \to F(A_n)
\]
in $\cat{B}$.  For example, given maps
\[
A_0 \toby{f_1} A_1 \toby{f_2} A_2 \toby{f_3} A_3 \toby{f_4} A_4
\]
in $\cat{A}$, we can construct maps
\[
\xymatrix@=10em{
F(A_0) 
\ar@<1ex>[r]^{F(f_4 f_3) F(f_2 f_1)}
\ar@<-1ex>[r]_{F(1_{A_4}) F(f_4) F(f_3 f_2) F(f_1)}     &
F(A_4)
}
\]
in $\cat{B}$, but the axioms imply that they are equal.  

\item   
\label{rmk:defn-ftr:comp}
We are familiar with the idea that structures and the structure-preserving
maps between them form a category (such as $\Grp$, $\Ring$, etc.).  In
particular, this applies to categories and functors: there is a category
$\CAT$%
%
\index{category!category of categories}%
\ntn{CAT}
%
whose objects are categories and whose maps are functors.

One part of this statement is that functors can be composed.%
%
\index{functor!composition of functors}
%
That is, given functors $\cat{A} \toby{F} \cat{B} \toby{G} \cat{C}$, there
arises a new functor $\cat{A} \toby{G \of F} \cat{C}$,%
%
\ntn{of-ftr}
%
defined in the obvious way.  Another is that for every category $\cat{A}$,
there is an identity%
%
\index{functor!identity}
%
functor $1_\cat{A}\from \cat{A} \to \cat{A}$.%
%
\ntn{id-ftr}
%
\end{enumerate}
\end{remarks}

\begin{examples} 
\label{egs:forgetful-functors}
Perhaps the easiest examples of functors are the so-called \demph{forgetful%
%
\index{functor!forgetful}
%
functors}.  (This is an informal term, with no precise definition.)  For
instance:
% 
\begin{enumerate}[(b)]
\item   
\label{eg:forgetful-groups}
There is a functor $U\from \Grp \to \Set$ defined as follows: if $G$ is a
group then $U(G)$ is the underlying%
%
\index{underlying}
%
set of $G$ (that is, its set of elements), and if $f\from G \to H$ is a
group homomorphism then $U(f)$ is the function $f$ itself.  So $U$ forgets
the group structure of groups and forgets that group homomorphisms are
homomorphisms.

\item   
\label{eg:forgetful-ring-vs}
Similarly, there is a functor $\Ring \to \Set$ forgetting the ring
structure on rings, and (for any field $k$) there is a functor $\Vect_k
\to \Set$ forgetting the vector space structure on vector spaces.

\item   
\label{eg:forgetful-part}
Forgetful functors do not have to forget \emph{all} the structure.  For
example, let $\Ab$%
%
\ntn{Ab}
%
be the category of abelian groups.  There is a functor $\Ring \to \Ab$ that
forgets the multiplicative structure, remembering just the underlying
additive group.  Or, let $\Mon$%
%
\ntn{Mon}
%
be the category of monoids.  There is a functor $U\from \Ring \to \Mon$
that forgets the additive structure, remembering just the underlying%
%
\index{underlying}
%
multiplicative monoid.  (That is, if $R$ is a ring then $U(R)$ is the set
$R$ made into a monoid via $\cdot$ and $1$.)

\item   
\label{eg:forgetful-ab}
There is an inclusion functor $U\from \Ab \to \Grp$ defined by $U(A) = A$
for any abelian group $A$ and $U(f) = f$ for any homomorphism $f$ of abelian
groups.  It forgets that abelian groups are abelian.
\end{enumerate}

The forgetful functors in examples
\bref{eg:forgetful-groups}--\bref{eg:forgetful-part} forget
\emph{structure} on the objects, but that of
example~\bref{eg:forgetful-ab} forgets a \emph{property}.  Nevertheless,
it turns out to be convenient to use the same word, `forgetful', in both
situations.

Although forgetting is a trivial operation, there are situations in which
it is powerful.  For example, it is a theorem that the order of any finite
field is a prime power.  An important step in the proof is to simply forget
that the field is a field, remembering only that it is a vector space over
its subfield $\{0, 1, 1 + 1, 1 + 1 + 1, \ldots\}$.
\end{examples}

\begin{examples}        
\label{egs:free-functors}
\demph{Free%
%
\index{functor!free}%
\index{free functor}
%
functors} are in some sense dual to forgetful functors (as we will see in
the next chapter), although they are less elementary.  Again, `free
functor' is an informal but useful term.

\begin{enumerate}[(b)]
\item 
\label{eg:free-group} 
Given any set $S$, one can build the \demph{free%
%
\index{group!free}
%
group} $F(S)$ on $S$.  This is a group containing $S$ as a subset and with
no further properties other than those it is forced to have, in a sense
made precise in Section~\ref{sec:adj-basics}.  Intuitively, the group
$F(S)$ is obtained from the set $S$ by adding just enough new elements that
it becomes a group, but without imposing any equations other than those
forced by the definition of group.

A little more precisely, the elements of $F(S)$ are formal expressions or
\demph{words}%
%
\index{word}
%
such as $x^{-4} y x^2 z y^{-3}$ (where $x, y, z \in S$).  Two such words
are seen as equal if one can be obtained from the other by the usual
cancellation rules, so that, for example, $x^3 x y$, $x^4 y$, and $x^2
y^{-1} y x^2 y$ all represent the same element of $F(S)$.  To multiply two
words, just write one followed by the other; for instance, $x^{-4} y x$
times $x z y^{-3}$ is $x^{-4} y x^2 z y^{-3}$.

This construction assigns to each set $S$ a group $F(S)$.  In fact, $F$ is
a functor: any map of sets $f\from S \to S'$ gives rise to a homomorphism
of groups $F(f)\from F(S) \to F(S')$.  For instance, take the map of sets
\[
f\from \{w, x, y, z\} \to \{u, v\}
\]
defined by $f(w) = f(x) = f(y) = u$ and $f(z) = v$.  This gives rise to a
homomorphism 
\[
F(f)\from F(\{w, x, y, z\}) \to F(\{u, v\}),
\]
which maps $x^{-4} y x^2 z y^{-3} \in F(\{w, x, y, z\})$ to 
\[
u^{-4} u u^2 v u^{-3} 
=
u^{-1} v u^{-3}
\in
F(\{u, v\}).
\]

\item 
\label{eg:free-ring} 
Similarly, we can construct the free commutative ring $F(S)$ on a set $S$,
giving a functor $F$ from $\Set$ to the category $\CRing$%
%
\ntn{CRing}
%
of commutative rings.  In fact, $F(S)$ is something familiar, namely, the
ring of polynomials%
%
\index{ring!polynomial}
%
over $\integers$ in commuting variables $x_s$ ($s \in S$).  (A polynomial
is, after all, just a formal expression built from the variables using
the ring operations $+$, $-$ and $\cdot$.)  For example, if $S$ is a
two-element set then $F(S) \iso \integers[x, y]$.

\item   
\label{eg:free-vs}
We can also construct the free%
%
\index{vector space!free}
%
vector space on a set.  Fix a field $k$.  The free functor $F\from \Set \to
\Vect_k$ is defined on objects by taking $F(S)$ to be a vector space with
basis $S$.  Any two such vector spaces are isomorphic; but it is perhaps
not obvious that there is any such vector space at all, so we have to
construct one.  Loosely, $F(S)$ is the set of all formal $k$-linear
combinations of elements of $S$, that is, expressions
\[
\sum_{s \in S} \lambda_s s
\]
where each $\lambda_s$ is a scalar and there are only finitely many values
of $s$ such that $\lambda_s \neq 0$.  (This restriction is imposed because
one can only take \emph{finite} sums in a vector space.)  Elements of
$F(S)$ can be added:
\[
\sum_{s \in S} \lambda_s s + \sum_{s \in S} \mu_s s 
=
\sum_{s \in S} (\lambda_s + \mu_s) s.
\]
There is also a scalar multiplication on $F(S)$:
\[
c \cdot \sum_{s \in S} \lambda_s s
=
\sum_{s \in S} (c \lambda_s) s
\]
($c \in k$).  In this way, $F(S)$ becomes a vector space.

To be completely precise and avoid talking about `expressions', we
can define $F(S)$ to be the set of all functions $\lambda\from S \to k$ such
that $\{ s \in S \such \lambda(s) \neq 0\}$ is finite.  (Think of such a
function $\lambda$ as corresponding to the expression $\sum_{s \in S}
\lambda(s) s$.)  To define addition on $F(S)$, we must define for each
$\lambda, \mu \in F(S)$ a sum $\lambda + \mu \in F(S)$; it is given by
\[
(\lambda + \mu)(s) = \lambda(s) + \mu(s)
\]
($s \in S$). Similarly, the scalar multiplication is given by $(c \cdot
\lambda)(s) = c\cdot \lambda(s)$ ($c \in k$, $\lambda \in F(S)$, $s \in
S$).  
\end{enumerate}

Rings and vector spaces have the special property that it is relatively
easy to write down an explicit formula for the free functor.  The case of
groups is much more typical.  For most types of algebraic structure,
describing the free functor requires as much fussy work as it does for
groups.  We return to this point in Example~\ref{egs:adjns-alg} and
Example~\ref{eg:gaft-free-alg} (where we see how to avoid the fussy work
entirely).
\end{examples}

\begin{examples}[Functors in algebraic topology]
\label{egs:functors:homo-homo}
%
\index{algebraic topology}   
%
Historically, some of the first examples of functors arose in algebraic
topology.  There, the strategy is to learn about a space by extracting data
from it in some clever way, assembling that data into an algebraic
structure, then studying the algebraic structure instead of the original
space.  Algebraic topology therefore involves many functors from categories
of spaces to categories of algebras.
% 
\begin{enumerate}[(b)]
\item 
Let $\Tp_*$%
%
\ntn{Top-star}
%
 be the category of topological spaces equipped with a
basepoint, together with the continuous basepoint-preserving maps.  There
is a functor $\pi_1\from \Tp_* \to \Grp$%
%
\ntn{pi-1}
%
assigning to each space $X$ with basepoint $x$ the fundamental%
%
\index{group!fundamental}
%
group $\pi_1(X, x)$ of $X$ at $x$.  (Some texts use the simpler notation
$\pi_1(X)$, ignoring the choice of basepoint.  This is more or less safe if
$X$ is path-connected, but strictly speaking, the basepoint should always
be specified.)

That $\pi_1$ is a functor means that it not only assigns to each
space-with-basepoint $(X, x)$ a group $\pi_1(X, x)$, but also assigns to
each basepoint-pre\-ser\-ving continuous map 
\[
f\from (X, x) \to (Y, y)
\]
a homomorphism 
\[
\pi_1(f)\from \pi_1(X, x) \to \pi_1(Y, y).
\]
Usually $\pi_1(f)$ is written as $f_*$.  The functoriality axioms say that $(g
\of f)_* = g_* \of f_*$ and $(1_{(X, x)})_* = 1_{\pi_1(X, x)}$.  

\item 
For each $n \in \nat$, there is a functor $H_n\from \Tp \to \Ab$%
%
\ntn{homology}
%
assigning to a space its $n$th homology%
%
\index{homology}
%
group (in any of several possible senses).
\end{enumerate}
\end{examples}

\begin{example}
Any system of polynomial%
%
\index{polynomial}%
\index{simultaneous equations}
%
equations such as
% 
\begin{align}
2x^2 + y^2 - 3z^2       &
= 
1
\label{eq:scheme-1}     \\
x^3 + x     &
=
y^2       
\label{eq:scheme-2}
\end{align}
% 
gives rise to a functor $\CRing \to \Set$.  Indeed, for each commutative
ring $A$, let $F(A)$ be the set of triples $(x, y, z) \in A \times A \times
A$ satisfying equations~\eqref{eq:scheme-1} and~\eqref{eq:scheme-2}.
Whenever $f\from A \to B$ is a ring homomorphism and $(x, y, z) \in F(A)$,
we have $(f(x), f(y), f(z)) \in F(B)$; so the map of rings $f\from A \to B$
induces a map of sets $F(f) \from F(A) \to F(B)$.  This defines a functor
$F\from \CRing \to \Set$.

In algebraic%
%
\index{algebraic geometry}
%
geometry, a \demph{scheme}%
%
\index{scheme}
%
is a functor $\CRing \to \Set$ with certain properties.  (This is not the
most common way of phrasing the definition, but it is equivalent.)  The
functor $F$ above is a simple example.
\end{example}

\begin{example}
\label{eg:ftrs-between-monoids}
Let $G$ and $H$ be monoids (or groups, if you prefer), regarded as
one-object%
%
\index{monoid!homomorphism of monoids}
%
categories $\cat{G}$ and $\cat{H}$.  A functor $F\from \cat{G} \to \cat{H}$
must send the unique object of $\cat{G}$ to the unique object of $\cat{H}$,
so it is determined by its effect on maps.  Hence, the functor $F \from
\cat{G} \to \cat{H}$ amounts to a function $F\from G \to H$ such that $F(g'
g) = F(g') F(g)$ for all $g', g \in G$, and $F(1) = 1$.  In other words, a
functor $\cat{G} \to \cat{H}$ is just a homomorphism $G \to H$.
\end{example}

\begin{example}
\label{eg:functor-action}
Let $G$ be a monoid,%
%
\index{monoid!action of}
%
regarded as a one-object category $\cat{G}$.  A functor $F\from \cat{G} \to
\Set$ consists of a set $S$ (the value of $F$ at the unique object of
$\cat{G}$) together with, for each $g \in G$, a function $F(g)\from S \to
S$, satisfying the functoriality axioms.  Writing $(F(g))(s) = g \cdot s$,
we see that the functor $F$ amounts to a set $S$ together with a function
\[
\begin{array}{ccc}
G \times S      &\to            &S      \\
(g, s)          &\mapsto        &g \cdot s      
\end{array}
\]
satisfying $(g' g) \cdot s = g' \cdot (g \cdot s)$ and $1 \cdot s = s$ for
all $g, g' \in G$ and $s \in S$.  In other words, a functor $\cat{G} \to
\Set$ is a set equipped with a left action by $G$: a \demph{left $G$-set},%
%
\index{G-set@$G$-set}
%
for short.

Similarly, a functor $\cat{G} \to \Vect_k$ is exactly a $k$-linear
representation%
%
\index{representation!group or monoid@of group or monoid!linear}
%
of $G$, in the sense of representation theory.  This can reasonably be
taken as the \emph{definition} of representation.
\end{example}

\begin{example}
\label{eg:functor-orders}
When $A$ and $B$ are (pre)ordered sets, a functor between the corresponding
categories is exactly an \demph{order-preserving%
%
\index{order-preserving}%
\index{map!order-preserving}
%
map}, that is, a function
$f\from A \to B$ such that $a \leq a' \implies f(a) \leq f(a')$.
Exercise~\ref{ex:functor-orders} asks you to verify this.
\end{example}

Sometimes we meet functor-like operations that reverse the arrows, with a
map $A \to A'$ in $\cat{A}$ giving rise to a map $F(A) \ot F(A')$ in
$\cat{B}$.  Such operations are called contravariant functors.

\begin{defn}    
\label{defn:contravariant}
Let $\cat{A}$ and $\cat{B}$ be categories.  A \demph{contravariant%
%
\index{functor!contravariant}%
\index{contravariant}
%
functor} from $\cat{A}$ to $\cat{B}$ is a functor $\cat{A}^\op \to
\cat{B}$.
\end{defn}

To avoid confusion, we write `a contravariant functor from $\cat{A}$ to
$\cat{B}$' rather than `a contravariant functor $\cat{A} \to \cat{B}$'.

Functors $\cat{C} \to \cat{D}$ correspond one-to-one with functors
$\cat{C}^\op \to \cat{D}^\op$, and $(\cat{A}^\op)^\op = \cat{A}$, so a
contravariant functor from $\cat{A}$ to $\cat{B}$ can also be described as
a functor $\cat{A} \to \cat{B}^\op$.  Which description we use is not
enormously important, but in the long run, the convention in
Definition~\ref{defn:contravariant} makes life easier.

An ordinary functor $\cat{A} \to \cat{B}$ is sometimes called a
\demph{covariant%
%
\index{functor!covariant}%
\index{covariant}
%
functor} from $\cat{A}$ to $\cat{B}$, for emphasis.

\begin{example}
\label{eg:contra-fn-spaces}
%
\index{topological space!functions on}%
\index{ring!functions@of functions}
%
We can tell a lot about a space by examining the functions on it.  The
importance of this principle in twentieth- and twenty-first-century
mathematics can hardly be exaggerated.

For example, given a topological space $X$, let $C(X)$%
%
\ntn{cts-ring}
%
be the ring of continuous real-valued functions on $X$.  The ring
operations are defined `pointwise':%
%
\index{pointwise}
%
for instance, if $p_1, p_2\from X \to \reals$ are continuous maps then the
map $p_1 + p_2 \from X \to \reals$ is defined by
\[
(p_1 + p_2)(x) = p_1(x) + p_2(x)
\]
($x \in X$).  A continuous map $f\from X \to Y$ induces a ring homomorphism
$C(f)\from C(Y) \to C(X)$, defined at $q \in C(Y)$ by taking $(C(f))(q)$ to be
the composite map
\[
X \toby{f} Y \toby{q} \reals.
\]
Note that $C(f)$ goes in the opposite direction from $f$.  After checking
some axioms (Exercise~\ref{ex:contra-fn-spaces}), we conclude that $C$ is a
contravariant functor from $\Tp$ to $\Ring$.

While this particular example will not play a large part in this text, it
is worth close attention.  It illustrates the important idea of a
structure whose elements are maps (in this case, a ring whose elements are
continuous functions).  The way in which $C$ becomes a functor, via
composition, is also important.  Similar constructions will be crucial in
later chapters.

For certain classes of space, the passage from $X$ to $C(X)$ loses no
information: there is a way of reconstructing the space $X$ from the ring
$C(X)$.  For this and related reasons, it is sometimes said that `algebra
is dual%
%
\index{duality!algebra--geometry}
%
to geometry'.
\end{example}

\begin{example}
\label{eg:fns-on-vs}
Let $k$ be a field.  For any two vector spaces $V$ and $W$ over $k$, there
is a vector space
\[
\HOM(V, W) 
=
\{ \text{linear maps } V \to W \}.%
%
\index{vector space!linear maps@of linear maps}
%
\ntn{HOM}
%
\]
The elements of this vector space are themselves maps, and the vector space
operations (addition and scalar multiplication) are defined pointwise, as
in the last example.

Now fix a vector space $W$.  Any linear map $f\from V \to V'$ induces a linear
map 
\[
f^*\from \HOM(V', W) \to \HOM(V, W),%
%
\ntn{dual-map}
%
\]
defined at $q\in \HOM(V', W)$ by taking $f^*(q)$ to be the composite map
\[
V \toby{f} V' \toby{q} W.
\]
This defines a functor
\[
\HOM(\dashbk, W)\from \Vect_k^\op \to \Vect_k.
\]
The symbol `$\dashbk$'%
%
\ntn{dashbk}
%
is a blank or placeholder, into which arguments can be inserted.  Thus, the
value of $\HOM(\dashbk, W)$ at $V$ is $\HOM(V, W)$.  Sometimes we use a
blank space%
%
\ntn{empty-blank}
%
instead of $\dashbk$, as in $\HOM(\hspace*{1em}, W)$.

An important special case is where $W$ is $k$, seen as a one-dimensional
vector space over itself.  The vector space $\HOM(V, k)$ is called the
\demph{dual}%
%
\index{vector space!functions on}%
\index{vector space!dual}%
\index{duality!vector spaces@for vector spaces}
%
of $V$, and is written as $V^*$.%
%
\ntn{dual-vs}
%
So there is a contravariant functor
\[
\blank^* = \HOM(\dashbk, k)\from \Vect_k^\op \to \Vect_k
\]
sending each vector space to its dual.
\end{example}

\begin{example}
For each $n\in\nat$, there is a functor $H^n\from  \Tp^\op
\to \Ab$%
%
\ntn{cohomology}
%
assigning to a space its $n$th cohomology%
%
\index{cohomology}
%
group.
\end{example}

\begin{example}
\label{eg:contra-functors:actions}
Let $G$ be a monoid, regarded as a one-object category $\cat{G}$.  A
functor $\cat{G}^\op \to \Set$ is a \emph{right} $G$-set,%
%
\index{monoid!action of}
%
for essentially the same reasons as in Example~\ref{eg:functor-action}.

That left actions are covariant functors and right actions are
contravariant functors is a consequence of a basic notational choice: we
write the value of a function $f$ at an element $x$ as $f(x)$, not $(x)f$.
\end{example}

Contravariant functors whose codomain is $\Set$ are important enough to have
their own special name.

\begin{defn}    
\label{defn:presheaf}
Let $\cat{A}$ be a category.  A \demph{presheaf}%
%
\index{presheaf|(}
%
on $\cat{A}$ is a functor $\cat{A}^\op \to \Set$.
\end{defn}

The name comes from the following special case.  Let $X$ be a topological
space.  Write $\oset(X)$%
%
\ntn{oset}
%
for the poset of open subsets of $X$, ordered by inclusion.  View
$\oset(X)$ as a category, as in
Example~\ref{egs:cats-as}\bref{eg:cats-as:orders}.  Thus, the objects of
$\oset(X)$ are the open subsets of $X$, and for $U, U' \in \oset(X)$, there
is one map $U \to U'$ if $U \sub U'$, and there are none otherwise.  A
\demph{presheaf} on the space $X$ is a presheaf on the category $\oset(X)$.
For example, given any space $X$, there is a presheaf $F$ on $X$ defined by
\[
F(U) = \{ \text{continuous functions } U \to \reals \}
%
\index{topological space!functions on}
%
\]
($U \in \oset(X)$) and, whenever $U \sub U'$ are open subsets of $X$, by
taking the map $F(U') \to F(U)$ to be restriction.  Presheaves, and a
certain class of presheaves called sheaves,%
%
\index{sheaf}
%
play an important role in modern geometry.%
%
\index{presheaf|)}
%

\subjectchange

We know very well that for \emph{functions} between \emph{sets}, it is
sometimes useful to consider special kinds of function such as injections,
surjections and bijections.  We also know that the notions of injection and
subset are related: for instance, whenever $B$ is a subset of $A$, there is
an injection $B \to A$ given by inclusion.  In this section and the next,
we introduce some similar notions for \emph{functors} between
\emph{categories}, beginning with the following definitions.

\begin{defn}
A functor $F\from  \cat{A} \to \cat{B}$ is \demph{faithful}%
%
\index{functor!faithful}%
\index{faithful}
%
(respectively, \demph{full})%
%
\index{functor!full}
%
if for each $A, A' \in \cat{A}$, the function
\[
\begin{array}{ccc}
\cat{A}(A, A')  &\to            &\cat{B}(F(A), F(A'))   \\
f               &\mapsto        &F(f)
\end{array}
\]
is injective (respectively, surjective).
\end{defn}

\begin{warning}
Note the roles of $A$ and $A'$ in the definition.  Faithfulness does
\emph{not} say that if $f_1$ and $f_2$ are distinct maps in $\cat{A}$ then
$F(f_1) \neq F(f_2)$ (Exercise~\ref{ex:faithful-not-inj}).
% 
\begin{figure}
\[
\cat{A}
\begin{array}{c}
\fbox{ 
\xymatrix{
\ &A\vphantom{F(A)} \ar@{.>}[d]&\ \\ &A'\vphantom{F(A')}
}
}
\end{array}
\qquad%
\toby{F}%
\qquad%
\begin{array}{c}
\fbox{ 
\xymatrix{
\ &F(A) \ar[d]^g&\  \\ & F(A')
}
}
\end{array}
\cat{B}
\]
\caption{Fullness and faithfulness.}
\label{fig:ff}
\end{figure}
% 
In the situation of Figure~\ref{fig:ff}, $F$ is faithful if for each $A$,
$A'$ and $g$ as shown, there is at most one dotted arrow that $F$ sends to
$g$.  It is full if for each such $A$, $A'$ and $g$, there is at least one
dotted arrow that $F$ sends to $g$.
\end{warning}

\begin{defn}
Let $\cat{A}$ be a category.  A \demph{subcategory}%
%
\index{subcategory!full}
%
$\cat{S}$ of $\cat{A}$ consists of a subclass $\ob(\cat{S})$ of
$\ob(\cat{A})$ together with, for each $S, S' \in \ob(\cat{S})$, a subclass
$\cat{S}(S, S')$ of $\cat{A}(S, S')$, such that $\cat{S}$ is closed under
composition and identities.  It is a \demph{full}%
%
\index{subcategory!full}
%
subcategory if $\cat{S}(S, S') = \cat{A}(S, S')$ for all $S, S'
\in \ob(\cat{S})$.
\end{defn}

A full subcategory therefore consists of a selection of the objects, with
all of the maps between them.  So, a full subcategory can be specified
simply by saying what its objects are.  For example, $\Ab$ is the full
subcategory of $\Grp$ consisting of the groups that are abelian.

Whenever $\cat{S}$ is a subcategory of a category $\cat{A}$, there is an
inclusion functor $I: \cat{S} \to \cat{A}$ defined by $I(S) = S$ and $I(f)
= f$.  It is automatically faithful, and it is full if and only if
$\cat{S}$ is a full subcategory.

\begin{warning}
The image%
%
\index{functor!image of}%
\index{image!functor@of functor}
%
of a functor need not be a subcategory.  For example, consider the functor
\[
\Bigl(
\begin{array}{c}
\xymatrix@1{A \ar[r]^-f&B\quad B' \ar[r]^-g&C}
\end{array}
\Bigr)
\qquad
\toby{F}
\qquad
\left(
\begin{array}{c}
\xymatrix{
        &Y \ar[dr]^q    &       \\
X \ar[ur]^p \ar[rr]_{qp} &&Z
}
\end{array}
\right)
\]
defined by $F(A) = X$, $F(B) = F(B') = Y$, $F(C) = Z$, $F(f) = p$, and $F(g) =
q$.  Then $p$ and $q$ are in the image of $F$, but $qp$ is not.
\end{warning}


\exs


\begin{question}
Find three examples of functors not mentioned above.
\end{question}


\begin{question}        
\label{ex:ftrs-pres-iso}
Show that functors preserve isomorphism.%
%
\index{isomorphism!preserved by functors}
%
That is, prove that if $F\from \cat{A} \to \cat{B}$ is a functor and $A, A'
\in \cat{A}$ with $A \iso A'$, then $F(A) \iso F(A')$.
\end{question}


\begin{question}
\label{ex:functor-orders}
Prove the assertion made in Example~\ref{eg:functor-orders}.  In other
words, given ordered sets $A$ and $B$, and denoting by $\cat{A}$ and
$\cat{B}$ the corresponding categories, show that a functor $\cat{A} \to
\cat{B}$ amounts to an order-preserving%
%
\index{order-preserving}%
\index{map!order-preserving}
%
map $A \to B$.
\end{question}


\begin{question}
Two categories $\cat{A}$ and $\cat{B}$ are \demph{isomorphic},%
%
\index{isomorphism!categories@of categories}%
\index{category!isomorphism of categories}
%
written as $\cat{A} \iso \cat{B}$,%
%
\ntn{iso-cat}
%
if they are isomorphic as objects of $\CAT$.  
% 
\begin{enumerate}[(b)]
\item 
Let $G$ be a group, regarded as a one-object%
%
\index{group!opposite}
%
category all of whose maps are isomorphisms.  Then its opposite $G^\op$ is
also a one-object category all of whose maps are isomorphisms, and can
therefore be regarded as a group too.  What is $G^\op$, in purely
group-theoretic terms?  Prove that $G$ is isomorphic to $G^\op$.

\item 
Find a monoid%
%
\index{monoid!opposite}
%
not isomorphic to its opposite.
\end{enumerate}
\end{question}


\begin{question}
Is there a functor $Z \from \Grp \to \Grp$ with the property that $Z(G)$ is
the centre%
%
\index{centre}
%
of $G$ for all groups $G$?
\end{question}


\begin{question}        
\label{ex:ftr-on-product}
Sometimes we meet functors whose domain is a product%
%
\index{category!product of categories}
%
$\cat{A} \times \cat{B}$ of categories.  Here you will show that such a
functor can be regarded as an interlocking pair of families of functors,
one defined on $\cat{A}$ and the other defined on $\cat{B}$.  (This is very
like the situation for bilinear and linear maps.)
% 
\begin{enumerate}[(b)]
\item   
\label{part:prod-ftr-compts} 
Let $F\from \cat{A} \times \cat{B} \to \cat{C}$ be a functor.  
Prove that for each $A \in \cat{A}$, there is a functor $F^A\from
\cat{B} \to \cat{C}$ defined on objects $B \in \cat{B}$ by $F^A(B) = F(A,
B)$ and on maps $g$ in $\cat{B}$ by $F^A(g) = F(1_A, g)$.  Prove that for
each $B \in \cat{B}$, there is a functor $F_B \from \cat{A} \to \cat{C}$
defined similarly.  

\item   
\label{part:prod-ftr-condns}
Let $F \from \cat{A} \times \cat{B} \to \cat{C}$ be a functor.  With
notation as in~\bref{part:prod-ftr-compts}, show that
the families of functors $(F^A)_{A \in \cat{A}}$ and $(F_B)_{B \in
  \cat{B}}$ satisfy the following two conditions:
% 
\begin{itemize}
\item 
if $A \in \cat{A}$ and $B \in \cat{B}$ then $F^A(B) = F_B(A)$;

\item 
if $f\from A \to A'$ in $\cat{A}$ and $g\from B \to B'$ in $\cat{B}$ then
$F^{A'}(g) \of F_B(f) = F_{B'}(f) \of F^A(g)$.
\end{itemize}

\item
Now take categories $\cat{A}$, $\cat{B}$ and $\cat{C}$, and take families
of functors $(F^A)_{A \in \cat{A}}$ and $(F_B)_{B \in \cat{B}}$ satisfying
the two conditions in~\bref{part:prod-ftr-condns}.  Prove that there is a
unique functor $F\from \cat{A} \times \cat{B} \to \cat{C}$ satisfying the
equations in~\bref{part:prod-ftr-compts}.  (`There is a unique functor'
means in particular that there \emph{is} a functor, so you have to prove
existence as well as uniqueness.)
\end{enumerate}
\end{question}


\begin{question}
\label{ex:contra-fn-spaces}
Fill in the details of Example~\ref{eg:contra-fn-spaces}, thus constructing
a functor $C\from \Tp^\op \to \Ring$.  
\end{question}


\begin{question}
\label{ex:faithful-not-inj}
Find an example of a functor $F\from \cat{A} \to \cat{B}$ such that $F$ is
faithful%
%
\index{functor!faithful}%
\index{faithful}
%
but there exist distinct maps $f_1$ and $f_2$ in $\cat{A}$ with $F(f_1) =
F(f_2)$.
\end{question}


\begin{question}
\begin{enumerate}[(b)]
\item
Of the examples of functors appearing in this section, which are faithful
and which are full?

\item 
Write down one example of a functor that is both full and faithful, one
that is full but not faithful, one that is faithful but not full, and one
that is neither.
\end{enumerate}
\end{question}


\begin{question}
\begin{enumerate}[(b)]
\item
What are the subcategories of an ordered set?  Which are full?

\item
What are the subcategories of a group?  (Careful!)  Which are full?
\end{enumerate}
\end{question}



\section{Natural transformations}
\label{sec:nts}


We now know about categories.  We also know about functors, which
are maps between categories.  Perhaps surprisingly, there is 
a further notion of `map between functors'.  Such maps are called natural
transformations.  This notion only applies when the functors have
the same domain and codomain: 
\[
\parpair{\cat{A}}{\cat{B}}{F}{G}.
\]

To see how this might work, let us consider a special case.  Let $\cat{A}$
be the discrete category
(Example~\ref{egs:cats-as}\bref{eg:cats-as:discrete}) whose objects are
the natural numbers $0, 1, 2$, \ldots.  A functor $F$ from $\cat{A}$ to
another category $\cat{B}$ is simply a sequence $(F_0, F_1, F_2, \ldots)$
of objects of $\cat{B}$.  Let $G$ be another functor from $\cat{A}$ to
$\cat{B}$, consisting of another sequence $(G_0, G_1, G_2, \ldots)$ of
objects of $\cat{B}$.  It would be reasonable to define a `map' from $F$ to
$G$ to be a sequence
\[
\Bigl(
F_0 \toby{\alpha_0} G_0, \ 
F_1 \toby{\alpha_1} G_1, \ 
F_1 \toby{\alpha_2} G_2, 
\ \ldots\ 
\Bigr)
\]
of maps in $\cat{B}$.  The situation can be depicted as follows:
\[
\cat{A} 
\begin{array}{c}
\fbox{
\xymatrix@C=2ex{0 & 1 & 2 & \cdots}
}
\end{array}
% 
\qquad
\qquad
% 
\begin{array}{c}
\fbox{
\xymatrix@C=2ex{
&
F_0 \ar[d]^{\alpha_0} &
F_1 \ar[d]^{\alpha_1} &
F_2 \ar[d]^{\alpha_2} &
{ } \ar@{}[d]^{\textstyle\cdots} &
&
\\
&
G_0 &
G_1 &
G_2 &
}
}
\end{array}
\cat{B} 
\]
(The right-hand diagram should not be understood too literally.  Some of
the objects $F_i$ or $G_i$ might be equal, and there might be much else in
$\cat{B}$ besides what is shown.)

This suggests that in the general case, a natural transformation between
functors $\parpairi{\cat{A}}{\cat{B}}{F}{G}$ should consist of maps
$\alpha_A\from F(A) \to G(A)$, one for each $A \in \cat{A}$.  In the
example above, the category $\cat{A}$ had the special property of not
containing any nontrivial maps.  In general, we demand some kind of
compatibility between the maps in $\cat{A}$ and the maps $\alpha_A$.

\begin{defn}
Let $\cat{A}$ and $\cat{B}$ be categories and let
$\parpairi{\cat{A}}{\cat{B}}{F}{G}$ be functors.  A \demph{natural%
%
\index{natural transformation}
%
transformation} $\alpha\from F \to G$ is a family $\Bigl( F(A)
\toby{\alpha_A} G(A) \Bigr)_{A \in \cat{A}}$%
%
\ntn{nt-comp}
%
of maps in $\cat{B}$ such that
for every map $A \toby{f} A'$ in $\cat{A}$, the square
% 
\begin{equation}        
\label{eq:nat}
\begin{array}{c}
\xymatrix{
F(A) \ar[r]^{F(f)} \ar[d]_{\alpha_A}    &
F(A') \ar[d]^{\alpha_{A'}}      \\
G(A) \ar[r]_{G(f)}      &
G(A')
} 
\end{array}
\end{equation}
% 
commutes.  The maps $\alpha_A$ are called the \demph{components}%
%
\index{component!natural transformation@of natural transformation}
%
of $\alpha$.
\end{defn}

\begin{remarks} 
\label{rmks:defn-nt}
\begin{enumerate}[(b)]
\item   
\label{rmk:defn-nt:loosely}
The definition of natural transformation is set up so that from each map $A
\toby{f} A'$ in $\cat{A}$, it is possible to construct exactly one%
%
\index{uniqueness!constructions@of constructions}
%
map $F(A) \to G(A')$ in $\cat{B}$.  When $f = 1_A$, this map is $\alpha_A$.
For a general $f$, it is the diagonal of the square~\eqref{eq:nat}, and
`exactly one' implies that the square commutes.

\item 
We write
\[
\xymatrix{
\cat{A} \rtwocell^F_G{\alpha} &\cat{B}
}%
%
\ntn{double-arrow}
%
\]
to mean that $\alpha$ is a natural transformation from $F$ to $G$.
\end{enumerate}
\end{remarks}

\begin{example}
Let $\cat{A}$ be a discrete%
%
\index{category!discrete!functor out of}
%
category, and let $F, G\from \cat{A} \to \cat{B}$ be functors.  Then $F$
and $G$ are just families $(F(A))_{A \in \cat{A}}$ and $(G(A))_{A \in
  \cat{A}}$ of objects of $\cat{B}$.  A natural transformation $\alpha\from
F \to G$ is just a family $\Bigl( F(A) \toby{\alpha_A} G(A) \Bigr)_{A \in
  \cat{A}}$ of maps in $\cat{B}$, as claimed above in the case $\ob \cat{A}
= \nat$.  In principle, this family must satisfy the naturality
axiom~\eqref{eq:nat} for every map $f$ in $\cat{A}$; but the only maps in
$\cat{A}$ are the identities, and when $f$ is an identity, this axiom holds
automatically.
\end{example}

\begin{example}
Recall from Examples~\ref{egs:cats-as} that a group (or more generally, a
monoid) $G$ can be regarded as a one-object%
%
\index{monoid!action of}
%
category.  Also recall from Example~\ref{eg:functor-action} that a functor
from the category $G$ to $\Set$ is nothing but a left $G$-set.  (Previously
we used $\cat{G}$ to denote the category corresponding to the group $G$;
from now on we use $G$%
%
\index{monoid!one-object category@as one-object category}
\ntn{group-one-obj}
%
to denote them both.)  Take two $G$-sets, $S$ and $T$.  Since $S$ and $T$
can be regarded as functors $G \to \Set$, we can ask: what is a natural
transformation
\[
\xymatrix{
G \rtwocell^S_T{\alpha} &\Set,
}
\]
in concrete terms?

Such a natural transformation consists of a single map in $\Set$ (since $G$
has just one object), satisfying some axioms.  Precisely, it is a function
$\alpha\from S \to T$ such that $\alpha(g\cdot s) = g\cdot \alpha(s)$ for
all $s \in S$ and $g \in G$.  (Why?)  In other words, it is just a map of
$G$-sets, sometimes called a \demph{$G$-equivariant}%
%
\index{equivariant}
%
map.
\end{example}

\begin{example}
Fix a natural number $n$.  In this example, we will see how `determinant%
%
\index{determinant}
%
of an $n \times n$ matrix' can be understood as a natural transformation.

For any commutative ring $R$, the $n \times n$ matrices with entries in $R$
form a monoid $M_n(R)$ under multiplication.  Moreover, any ring
homomorphism $R \to S$ induces a monoid homomorphism $M_n(R) \to M_n(S)$.
This defines a functor $M_n\from \CRing \to \Mon$ from the category of
commutative rings to the category of monoids.

Also, the elements of any ring $R$ form a monoid $U(R)$ under multiplication,
giving another functor $U\from \CRing \to \Mon$.

Now, every $n \times n$ matrix $X$ over a commutative ring $R$
has a determinant ${\det}_R(X)$, which is an element of $R$.
Familiar properties of determinant~--
\[
{\det}_R(XY) = {\det}_R(X) {\det}_R(Y),
\qquad
{\det}_R(I) = 1
\]
-- tell us that for each $R$, the function ${\det}_R\from M_n(R) \to U(R)$
is a monoid homomorphism.  So, we have a family of maps
\[
\Bigl( M_n(R) \toby{\det_R} U(R) \Bigr)_{R \in \CRing},
\]
and it makes sense to ask whether they define a natural transformation
\[
\xymatrix{
\CRing \rrtwocell^{M_n}_U{\hspace{.8em}\det} &&\Mon.
}
\]
Indeed, they do.  That the naturality squares commute (check!)\ reflects
the fact that determinant is defined in the same way for all rings.  We do
not use one definition of determinant for one ring and a different
definition for another ring.  Generally speaking, the naturality
axiom~\eqref{eq:nat} is supposed to capture the idea that the family
$(\alpha_A)_{A \in \cat{A}}$ is defined in a uniform way across all $A \in
\cat{A}$.
\end{example}

\begin{constn}  
\label{constn:comp-nat}
Natural transformations are a kind of map, so we would expect to be able to
compose%
%
\index{natural transformation!composition of}
%
them.  We can.  Given natural transformations
% 
\vspace{-2ex}
\[
\xymatrix@C+1.5em{
\cat{A} 
\ruppertwocell<8>^F{\alpha}
\ar[r]|G 
\rlowertwocell<-8>_H{\beta}
&
\cat{B}
},
\]
\vspace{-3ex}\\
% 
there is a composite natural transformation
\[
\xymatrix@C+1.5em{
\cat{A} \rtwocell^F_H{\hspace{.8em}\beta\of\alpha} &\cat{B}
}%
%
\ntn{of-nt}
%
\]
defined by $(\beta\of\alpha)_A = \beta_A \of \alpha_A$ for all $A \in
\cat{A}$.  There is also an identity%
%
\index{natural transformation!identity}
%
natural transformation
\[
\xymatrix@C+1em{
\cat{A} \rtwocell^F_F{\hspace{.5em}1_F} &\cat{B}
}%
%
\ntn{id-nt}
%
\]
on any functor $F$, defined by $(1_F)_A = 1_{F(A)}$.  So for any two
categories $\cat{A}$ and $\cat{B}$, there is a category whose objects are
the functors from $\cat{A}$ to $\cat{B}$ and whose maps are the natural
transformations between them.  This is called the \demph{functor%
%
\index{functor!category}
%
category} from $\cat{A}$ to $\cat{B}$, and written as
$\ftrcat{\cat{A}}{\cat{B}}$%
%
\ntn{ftr-cat-bkts}
%
 or
$\cat{B}^\cat{A}$.%
%
\ntn{ftr-cat-power}
%
\end{constn}

\begin{example}        
Let $2$%
%
\ntn{two-disc-cat}
%
be the discrete%
%
\index{category!discrete!functor out of}
%
category with two objects.  A functor from $2$ to a category $\cat{B}$ is a
pair of objects of $\cat{B}$, and a natural transformation is a pair of
maps.  The functor category $\ftrcat{2}{\cat{B}}$ is therefore isomorphic
to the product category $\cat{B} \times \cat{B}$
(Construction~\ref{constn:prod-cat}).  This fits well with the alternative
notation $\cat{B}^2$ for the functor category.
\end{example}

\begin{example} 
Let $G$ be a monoid.%
%
\index{monoid!action of}
%
Then $\ftrcat{G}{\Set}$%
%
\ntn{GSet}
%
is the category of left $G$-sets, and $\ftrcat{G^\op}{\Set}$ is the
category of right $G$-sets (Example~\ref{eg:contra-functors:actions}).
\end{example}

\begin{example}
\label{eg:ftr-cats-orders}
Take ordered%
%
\index{ordered set}
%
sets $A$ and $B$, viewed as categories (as in
Example~\ref{egs:cats-as}\bref{eg:cats-as:orders}).  Given order-preserving
maps $\parpairi{A}{B}{f}{g}$, viewed as functors (as in
Example~\ref{eg:functor-orders}), there is at most one natural
transformation
\[
\xymatrix{
A \rtwocell^f_g &B,
}
\]
and there is one if and only if $f(a) \leq g(a)$ for all $a \in A$.  (The
naturality axiom~\eqref{eq:nat} holds automatically, because in an ordered
set, all diagrams commute.)  So $\ftrcat{A}{B}$ is an ordered set too; its
elements are the order-preserving maps from $A$ to $B$, and $f \leq g$ if
and only if $f(a) \leq g(a)$ for all $a \in A$.
\end{example}

Everyday phrases such as `\emph{the}%
%
\index{uniqueness}
%
cyclic group of order $6$' and `\emph{the} product of two spaces' reflect
the fact that given two isomorphic objects of a category, we usually
neither know nor care%
% 
\label{p:care}
% 
whether they are actually equal.  This is enormously important.  

In particular, the lesson applies when the category concerned is a functor
category.  In other words, given two functors $F, G\from \cat{A} \to
\cat{B}$, we usually do not care whether they are literally equal.
(Equality would imply that the objects $F(A)$ and $G(A)$ of $\cat{B}$ were
equal for all $A \in \cat{A}$, a level of detail in which we have just
declared ourselves to be uninterested.)  What really matters is whether
they are naturally isomorphic.

\begin{defn}    
\label{defn:nat-iso}
Let $\cat{A}$ and $\cat{B}$ be categories.  A \demph{natural%
%
\index{isomorphism!natural}
%
isomorphism} between functors from $\cat{A}$ to $\cat{B}$ is an isomorphism
in $\ftrcat{\cat{A}}{\cat{B}}$.
\end{defn}

An equivalent form of the definition is often useful:

\begin{lemma}   
\label{lemma:nat-iso-compts}
Let $\xymatrix@1{\cat{A}\rtwocell^F_G{\alpha} &\cat{B}}$ be a natural
transformation.  Then $\alpha$ is a natural isomorphism if and only if
$\alpha_A\from F(A) \to G(A)$ is an isomorphism for all $A \in \cat{A}$.
\end{lemma}

\begin{pf}
Exercise~\ref{ex:nat-iso-compts}.
\end{pf}

Of course, we say that functors $F$ and $G$ are \demph{naturally
isomorphic} if there exists a natural isomorphism from $F$ to $G$.  Since
natural isomorphism is just isomorphism in a particular category (namely,
$\ftrcat{\cat{A}}{\cat{B}}$), we already have notation for this: $F \iso
G$.%
%
\ntn{iso-ftr}
%

\begin{defn}	
\label{defn:nat-in}
Given functors $\parpair{\cat{A}}{\cat{B}}{F}{G}$, we say that
% 
\[
F(A) \iso G(A) \text{ \demph{naturally in} } A 
%
\index{naturally}
%
\]
if $F$ and $G$ are naturally isomorphic.  
\end{defn}

This alternative terminology can be understood as follows.  If $F(A) \iso
G(A)$ naturally in $A$ then certainly $F(A) \iso G(A)$ for each individual
$A$, but more is true: we can choose isomorphisms $\alpha_A\from F(A) \to
G(A)$ in such a way that the naturality axiom~\eqref{eq:nat} is satisfied.

\begin{example}
Let $F, G\from \cat{A} \to \cat{B}$ be functors from a discrete%
%
\index{category!discrete!functor out of}
%
category $\cat{A}$ to a category $\cat{B}$.  Then $F \iso G$ if and only if
$F(A) \iso G(A)$ for all $A \in \cat{A}$.

So in \emph{this} case, $F(A) \iso G(A)$ naturally in $A$ if and only if
$F(A) \iso G(A)$ for all $A$.  But this is only true because $\cat{A}$ is
discrete.  In general, it is emphatically false.  There are many examples
of categories and functors $\parpairi{\cat{A}}{\cat{B}}{F}{G}$ such that
$F(A) \iso G(A)$ for all $A \in \cat{A}$, but not \emph{naturally} in $A$.
Exercise~\ref{ex:species} gives an example from combinatorics.
\end{example}

\begin{example}
Let $\FDVect$%
%
\ntn{FDVect}
%
be the category of finite-dimensional vector spaces over some field $k$.
The dual%
%
\index{vector space!dual}%
\index{duality!vector spaces@for vector spaces}
%
vector space construction defines a contravariant functor from $\FDVect$ to
itself (Example~\ref{eg:fns-on-vs}), and the double dual construction
therefore defines a covariant functor from $\FDVect$ to itself.

Moreover, we have for each $V \in \FDVect$ a canonical isomorphism
$\alpha_V\from V \to V^{**}$.  Given $v \in V$, the element $\alpha_V(v)$
of $V^{**}$ is `evaluation%
%
\index{evaluation}
%
at $v$'; that is, $\alpha_V(v)\from V^* \to k$ maps $\phi \in V^*$ to
$\phi(v) \in k$.  That $\alpha_V$ is an isomorphism is a standard result in
the theory of finite-dimensional vector spaces.

This defines a natural transformation
\[
\xymatrix{
\FDVect 
\rtwocell<5>^{1_\FDVect}_{\blank^{**}}{\alpha} &
\FDVect
}
\]
from the identity functor to the double dual functor.  By
Lemma~\ref{lemma:nat-iso-compts}, $\alpha$ is a natural isomorphism.  So
$1_\FDVect \iso \blank^{**}$.  Equivalently, in the language of
Definition~\ref{defn:nat-in}, $V \iso V^{**}$ naturally in $V$.

This is one of those occasions on which category theory makes an intuition
precise.  In some informal sense, evident before you learn anything about
category theory, the isomorphism between a finite-dimensional vector space
and its double dual is `natural' or `canonical': no arbitrary choices are
needed in order to define it.  In contrast, to specify an isomorphism
between $V$ and its single dual $V^*$, we need to make an arbitrary choice
of basis, and the isomorphism really does depend on the basis that we
choose.
\end{example}

In the example on vector spaces, the word \demph{canonical}%
%
\index{canonical}
%
was used.  It is an informal word, meaning something like `God-given' or
`defined without making arbitrary choices'.  For example, for any two sets
$A$ and $B$, there is a canonical bijection $A \times B \to B \times A$
defined by $(a, b) \mapsto (b, a)$, and there is a canonical function $A
\times B \to A$ defined by $(a, b) \mapsto a$.  But the function $B \to A$
defined by `choose an element $a_0 \in A$ and send everything to $a_0$' is
not canonical, because the choice of $a_0$ is arbitrary.

\subjectchange

The concept of natural isomorphism leads unavoidably to another central
concept: equivalence of categories.

%
\index{sameness|(}
%
Two elements of a set are either equal or not.  Two objects of a category
can be equal, not equal but isomorphic, or not even isomorphic.  As
explained before Definition~\ref{defn:nat-iso}, the notion of equality
between two objects of a category is unreasonably strict; it is usually
isomorphism that we care about.  So:
% 
\begin{itemize}
\item
the right notion of sameness of two elements of a set is
equality;

\item
the right notion of sameness of two objects of a category is
isomorphism.
\end{itemize}
% 
When applied to a functor category $\ftrcat{\cat{A}}{\cat{B}}$, the second
point tells us that:
% 
\begin{itemize}
\item
the right notion of sameness of two functors $\cat{A} \parpairu
\cat{B}$ is natural isomorphism. 
\end{itemize}
% 
But what is the right notion of sameness of two \emph{categories}?
Isomorphism is unreasonably strict, as if $\cat{A} \iso \cat{B}$ then
there are functors
% 
\begin{equation}        
\label{eq:oppair}
\oppair{\cat{A}}{\cat{B}}{F}{G}
\end{equation}
% 
such that 
% 
\begin{equation}
\label{eq:inverse-functors}
G \of F = 1_\cat{A}
\qquad
\text{and}
\qquad
F \of G = 1_\cat{B}, 
\end{equation}
% 
and we have just seen that the notion of equality between functors is too
strict.  The most useful notion of sameness of categories, called
`equivalence', is looser than isomorphism.  To obtain the definition, we
simply replace the unreasonably strict equalities
in~\eqref{eq:inverse-functors} by isomorphisms.  This gives
\[
G \of F \iso 1_\cat{A}
\qquad
\text{and}
\qquad
F \of G \iso 1_\cat{B}.
%
\index{sameness|)}
%
\]

\begin{defn}    
\label{defn:eqv}
An \demph{equivalence}%
%
\index{equivalence of categories}%
\index{category!equivalence of categories}
%
between categories $\cat{A}$ and $\cat{B}$ consists
of a pair~\eqref{eq:oppair} of functors together with natural isomorphisms
\[
\eta\from 1_\cat{A} \to G \of F,
\qquad
\epsln\from F \of G \to 1_\cat{B}.
\]
If there exists an equivalence between $\cat{A}$ and $\cat{B}$, we say that
$\cat{A}$ and $\cat{B}$ are \demph{equivalent}, and write $\cat{A} \eqv
\cat{B}$.%
%
\ntn{eqv}
%
We also say that the functors $F$ and $G$ are \demph{equivalences}.
\end{defn}

The directions of $\eta$ and $\epsln$ are not very important, since they
are isomorphisms anyway.  The reason for this particular choice will become
apparent when we come to discuss adjunctions (Section~\ref{sec:adj-units}).

\begin{warning}
The symbol $\iso$ is used for isomorphism of objects of a category, and in
particular for isomorphism of categories (which are objects of $\CAT$).
The symbol $\eqv$ is used for equivalence of categories.  At least, this is
the convention used in this book and by most category theorists, although
it is far from universal in mathematics at large.
\end{warning}

There is a very useful alternative characterization of those functors that
are equivalences.  First, we need a definition.

\begin{defn} 
A functor $F\from \cat{A} \to \cat{B}$ is \demph{essentially%
%
\index{essentially surjective on objects}%
\index{functor!essentially surjective on objects}
%
surjective on objects} if for all $B \in \cat{B}$, there exists $A \in
\cat{A}$ such that $F(A) \iso B$.
\end{defn}

\begin{propn}   
\label{propn:eqv-ffeso}
A functor is an equivalence if and only if it is full, faithful and
essentially surjective on objects.
\end{propn}

\begin{pf}
Exercise~\ref{ex:eqv-ffeso}.
\end{pf}

This result can be compared to the theorem that every bijective group
homomorphism is an isomorphism (that is, its inverse is also a
homomorphism), or that a natural transformation whose components are
isomorphisms is itself an isomorphism (Lemma~\ref{lemma:nat-iso-compts}).
Those two results are useful because they allow us to show that a map is an
isomorphism without directly constructing an inverse.
Proposition~\ref{propn:eqv-ffeso} provides a similar service, enabling us
to prove that a functor $F$ is an equivalence without actually constructing
an `inverse' $G$, or indeed an $\eta$ or an $\epsln$ (in the notation of
Definition~\ref{defn:eqv}).

A corollary of Proposition~\ref{propn:eqv-ffeso} invites us to view full
and faithful%
%
\index{functor!full and faithful}
%
functors as, essentially, inclusions of full subcategories:

\begin{cor}     
\label{cor:ff-emb}
Let $F\from \cat{C} \to \cat{D}$ be a full and faithful functor.  Then
$\cat{C}$ is equivalent to the full subcategory $\cat{C}'$ of $\cat{D}$
whose objects are those of the form $F(C)$ for some $C \in \cat{C}$.
\end{cor}

\begin{pf}
The functor $F'\from \cat{C} \to \cat{C}'$ defined by $F'(C) = F(C)$ is
full and faithful (since $F$ is) and essentially surjective on objects (by
definition of $\cat{C}'$).
\end{pf}

This result is true, with the same proof, whether we interpret `of the form
$F(C)$' to mean `equal to $F(C)$' or `isomorphic to $F(C)$'.

\begin{example}	
\label{eg:equivs-skellish}
Let $\cat{A}$ be any category, and let $\cat{B}$ be any full subcategory
containing at least one object from each isomorphism class of $\cat{A}$.
Then the inclusion functor $\cat{B} \incl \cat{A}$ is faithful (like any
inclusion of subcategories), full, and essentially surjective on objects.
Hence $\cat{B} \eqv \cat{A}$.

So if we take a category and remove some (but not all) of the objects in
each isomorphism class, the slimmed-down%
%
\index{category!slimmed-down}
%
version is equivalent to the original.  Conversely, if we take a category
and throw in some more objects, each of them isomorphic to one of the
existing objects, it makes no difference: the new, bigger, category is
equivalent to the old one.

For example, let $\FinSet$%
%
\ntn{FinSet}
%
be the category of finite%
%
\index{set!finite}
%
sets and functions between them.  For each natural number $n$, choose a set
$\lwr{n}$ with $n$ elements, and let $\cat{B}$ be the full subcategory of
$\FinSet$ with objects $\lwr{0}, \lwr{1}$, \ldots.  Then $\cat{B} \eqv
\FinSet$, even though $\cat{B}$ is in some sense much smaller than
$\FinSet$.
\end{example}

\begin{example}
\label{eg:equivs-mon}
In Example~\ref{egs:cats-as}\bref{eg:cats-as:monoids}, we saw that monoids
are essentially the same thing as one-object%
%
\index{monoid!one-object category@as one-object category}
%
categories.  With the definition of equivalence in hand, we are nearly
ready to make this statement precise.  We are missing some set-theoretic
language, and we will return to this result once we have that language
(Example~\ref{eg:mon-one-obj-eqv}), but the essential point can be stated
now.

Let $\cat{C}$ be the full subcategory of $\CAT$ whose objects are the
one-object categories.  Let $\Mon$ be the category of monoids.  Then
$\cat{C} \eqv \Mon$.  To see this, first note that given any object $A$ of
any category, the maps $A \to A$ form a monoid under composition (at least,
subject to some set-theoretic restrictions).  There is, therefore, a
canonical functor $F: \cat{C} \to \Mon$ sending a one-object category to
the monoid of maps from the single object to itself.  This functor $F$ is
full and faithful (by Example~\ref{eg:ftrs-between-monoids}) and
essentially surjective on objects.  Hence $F$ is an equivalence.
\end{example}

\begin{example}
An equivalence of the form $\cat{A}^\op \eqv \cat{B}$ is sometimes called a
\demph{duality}%
%
\index{duality}
%
between $\cat{A}$ and $\cat{B}$.  One says that $\cat{A}$ is \demph{dual}
to $\cat{B}$.  There are many famous dualities in which $\cat{A}$ is a
category of algebras and $\cat{B}$ is a category of spaces; recall the
slogan `algebra is dual%
%
\index{duality!algebra--geometry}
%
to geometry' from Example~\ref{eg:contra-fn-spaces}.

Here are some quite advanced examples, well beyond the scope of this book.
% 
\begin{itemize}
\item 
Stone duality:%
%
\index{duality!Stone}
%
the category of Boolean%
%
\index{Boolean algebra}
%
algebras is dual to the category of totally disconnected compact Hausdorff
spaces.

\item 
Gelfand--Naimark duality:%
%
\index{duality!Gelfand--Naimark}
%
the category of commutative unital $C^*$-algebras%
%
\index{C-algebra@$C^*$-algebra}
%
is dual to the category of compact Hausdorff spaces.  ($C^*$-algebras are
certain algebraic structures important in functional analysis.)

\item 
Algebraic geometers%
%
\index{algebraic geometry}
%
have several notions of `space', one of which is `affine variety'.%
%
\index{variety}
%
Let $k$ be an algebraically closed field.  Then the category of affine
varieties over $k$ is dual to the category of finitely generated
$k$-algebras with no nontrivial nilpotents.

\item 
Pontryagin duality:%
%
\index{duality!Pontryagin}
%
the category of locally compact abelian topological%
%
\index{topological group}%
\index{group!topological}\linebreak
%
groups is dual to itself.  As the words `topological group' suggest, both
sides of the duality are algebraic \emph{and} geometric.  Pontryagin
duality is an abstraction of the properties of the Fourier%
%
\index{Fourier analysis}
%
transform.
\end{itemize}
\end{example}


\begin{example}
It is rarely useful to consider a category of structured objects in
which the maps do not respect that structure.  For instance, let $\cat{A}$
be the category whose objects are groups%
%
\index{group!non-homomorphisms of groups}
%
and whose maps are \emph{all} functions between them, not necessarily
homomorphisms.  Let $\Set_{\neq\emptyset}$ be the category of nonempty
sets.  The forgetful functor $U\from \cat{A} \to \Set_{\neq\emptyset}$ is
full and faithful.  It is a (not profound) fact that every nonempty set can
be given at least one group structure, so $U$ is essentially surjective on
objects.  Hence $U$ is an equivalence.  This implies that the category
$\cat{A}$, although defined in terms of groups, is really just the category
of nonempty sets.
\end{example}

\begin{remarks} 
\label{rmks:2-cat-CAT}
Here is a kind of review of the chapter so far.  We have defined:
% 
\begin{itemize}
\item 
categories (Section~\ref{sec:cats});

\item 
functors between categories (Section~\ref{sec:ftrs});

\item 
natural transformations between functors (Section~\ref{sec:nts});

\item 
composition of functors 
\[
\cdot \to \cdot \to \cdot
\]
and the identity functor on any category
(Remark~\ref{rmks:defn-ftr}\bref{rmk:defn-ftr:comp});

\item 
composition of natural transformations%
%
\index{natural transformation!composition of|(}
%
\vspace{-3ex}
\[
\xymatrix{
\cdot
\ruppertwocell
\ar[r]
\rlowertwocell
&
\cdot
}
\]
\vspace{-4ex}\\ 
%
and the identity natural transformation on any functor
(Construction~\ref{constn:comp-nat}).
\end{itemize}
% 
This composition of natural transformations is sometimes called
\demph{vertical%
%
\index{composition!vertical}
%
composition}.  There is also \demph{horizontal%
%
\index{composition!horizontal} 
%
composition}, which takes natural transformations
\[
\xymatrix@C+.5em{
\cat{A} \rtwocell<4>^F_G{\alpha}   &
\cat{A}' \rtwocell<4>^{F'}_{G'}{\alpha'}   &
\cat{A}''
}
\]
and produces a natural transformation
\[
\xymatrix@C+.5em{
\cat{A} \rtwocell<4>^{F' \of F}_{G' \of G} &\cat{A}'',
}
\]
traditionally written as $\alpha' * \alpha$.%
%
\ntn{horiz-comp}
%
The component of $\alpha' * \alpha$ at $A \in \cat{A}$ is defined to be
the diagonal of the naturality square
\[
\xymatrix{
F'(F(A)) \ar[r]^{F'(\alpha_A)} \ar[d]_{\alpha'_{F(A)}}  &
F'(G(A)) \ar[d]^{\alpha'_{G(A)}}        \\
G'(F(A)) \ar[r]_{G'(\alpha_A)}  &
G'(G(A)).
}
\]
In other words, $(\alpha' * \alpha)_A$ can be defined as either 
$\alpha'_{G(A)} \of F'(\alpha_A)$ or $G'(\alpha_A) \of \alpha'_{F(A)}$; it
makes no difference which, since they are equal.  

The special%
% 
\label{p:special-cases}
%
cases of horizontal composition where either $\alpha$ or $\alpha'$ is an
identity are especially important, and have their own notation.  Thus,
\[
\xymatrix@C+.5em{
\cat{A}
\ar[r]^F        &
\cat{A}'
\rtwocell^{F'}_{G'}{\hspace{.3em}\alpha'}    &
\cat{A}''
}
\qquad
\text{gives rise to}
\qquad
\xymatrix@C+1.5em{
\cat{A} \rtwocell<4>^{F'\of F}_{G'\of F}{\hspace{.8em}\alpha' F}      &\cat{A}''
}%
%
\ntn{whisker-right}
%
\]
where $(\alpha' F)_A = \alpha'_{F(A)}$, and 
\[
\xymatrix@C+.5em{
\cat{A}
\rtwocell^F_G{\alpha}   &
\cat{A}'
\ar[r]^{F'}     &
\cat{A}''
}
\qquad
\text{gives rise to}
\qquad
\xymatrix@C+1.5em{
\cat{A} \rtwocell<4>^{F'\of F}_{F' \of G}{\hspace{.8em}F' \alpha} &
\cat{A}''
}%
%
\ntn{whisker-left}
%
\]
where $(F'\alpha)_A = F'(\alpha_A)$.  

Vertical and horizontal composition interact well: natural transformations
\[
\xymatrix{
\cat{A} 
\ruppertwocell<8>^F{\alpha}
\ar[r]|G
\rlowertwocell<-8>_H{\beta} &
\cat{A}'
\ruppertwocell<8>^{F'}{\hspace{.2em}\alpha'}
\ar[r]|{G'}
\rlowertwocell<-8>_{H'}{\hspace{.2em}\beta'} &
\cat{A}''
}
\]
obey the \demph{interchange law},%
%
\index{interchange law}
%
\[
(\beta' \of \alpha') * (\beta \of \alpha)
=
(\beta' * \beta) \of (\alpha' * \alpha)
\from
F' \of F \to H' \of H.
\]
As usual, a statement on composition is accompanied by a statement on
identities: $1_{F'} * 1_F = 1_{F' \of F}$ too.

All of this enables us to construct, for any categories $\cat{A}$,
$\cat{A}'$ and $\cat{A}''$, a functor
\[
\ftrcat{\cat{A}'}{\cat{A}''}
\times
\ftrcat{\cat{A}}{\cat{A}'}
\to
\ftrcat{\cat{A}}{\cat{A}''},
%
\index{functor!category}
%
\]
given on objects by $(F', F) \mapsto F' \of F$ and on maps by $(\alpha',
\alpha) \mapsto \alpha' * \alpha$.  In particular, if $F' \iso G'$ and $F
\iso G$ then $F' \of F \iso G' \of G$, since functors preserve isomorphism
(Exercise~\ref{ex:ftrs-pres-iso}).

(The existence of this functor is similar to the fact that \emph{inside} a
category $\cat{C}$, we have, for any objects $A$, $A'$ and $A''$, a
funct\emph{ion}
\[
\cat{C}(A', A'') \times \cat{C}(A, A') \to \cat{C}(A, A''),
\]
given by $(f', f) \mapsto f' \of f$.)

The diagrams above contain not only objects (0-dimensional) and arrows
$\to$ (1-dimensional), but also double arrows $\Rightarrow$ sweeping out
2-dimensional regions between arrows.  What we are implicitly doing is
called 2-category%
%
\index{two-category@2-category}
%
theory.  There is a 2-category%
%
\index{category!two-category of categories@2-category of categories}
%
of categories, functors and natural transformations, whose anatomy we have
just been describing.  If we are really serious about categories, we have
to get serious about 2-categories.  And if we are really serious about
2-categories, we have to get serious about 3-categories\ldots%
%
\index{n-category@$n$-category}
%
and before we know it, we are studying $\infty$-categories.%
%
\index{infinity-category@$\infty$-category}
%
But in this book, we climb no higher than the first rung or two of this
infinite ladder.%
%
\index{natural transformation!composition of|)}
%

\end{remarks}


\exs


\begin{question}
Find three examples of natural transformations not mentioned above.
\end{question}


\begin{question}        
\label{ex:nat-iso-compts}
Prove Lemma~\ref{lemma:nat-iso-compts}.
\end{question}


\begin{question}
Let $\cat{A}$ and $\cat{B}$ be categories.  Prove that
$\ftrcat{\cat{A}^\op}{\cat{B}^\op} \iso \ftrcat{\cat{A}}{\cat{B}}^\op$.
\end{question}


\begin{question}
Let $A$ and $B$ be sets, and denote by $B^A$ the set of functions from $A$
to $B$.  Write down:
% 
\begin{enumerate}[(b)]
\item 
a canonical function $A \times B^A \to B$;%
%
\index{canonical}
%

\item 
a canonical function $A \to B^{(B^A)}$.
\end{enumerate}
% 
(Although in principle there could be many such canonical functions,
in both these cases there is only one.)
\end{question}


\begin{question}        
\label{ex:nat-iso-on-product}
Here we consider natural transformations between functors whose domain
is a product%
%
\index{category!product of categories}
%
category $\cat{A} \times \cat{B}$.  Your task is to show that naturality in
two variables simultaneously is equivalent to naturality in each variable
separately.

Take functors $F, G\from \cat{A} \times \cat{B} \to \cat{C}$.  For each $A
\in \cat{A}$, there are functors $F^A, G^A\from \cat{B} \to \cat{C}$, as in
Exercise~\ref{ex:ftr-on-product}.  Similarly, for each $B\in\cat{B}$,
there are functors $F_B, G_B\from \cat{A} \to \cat{C}$.

Let $\bigl(\alpha_{A, B}\from F(A, B) \to G(A, B)\bigr)_{A \in \cat{A}, B
  \in \cat{B}}$ be a family of maps.  Show that this family is a natural
transformation $F \to G$ if and only if it satisfies the following two
conditions: 
% 
\begin{itemize}
\item 
for each $A \in \cat{A}$, the family $\bigl(\alpha_{A, B}\from F^A(B) \to
G^A(B)\bigr)_{B \in \cat{B}}$ is a natural transformation $F^A \to G^A$;

\item 
for each $B \in \cat{B}$, the family $\bigl(\alpha_{A, B}\from F_B(A) \to
G_B(A)\bigr)_{A \in \cat{A}}$ is a natural transformation $F_B \to G_B$.
\end{itemize}
\end{question}


\begin{question}
Let $G$ be a group.%
%
\index{group!isomorphism of elements of}
%
For each $g \in G$, there is a unique homomorphism
$\phi\from \integers \to G$%
%
\index{Z@$\integers$ (integers)!group@as group}
%
satisfying $\phi(1) = g$.  Thus, elements of $G$ are essentially the same
thing as homomorphisms $\integers \to G$.  When groups are regarded as
one-object categories, homomorphisms $\integers \to G$ are in turn the same
as functors $\integers \to G$.  Natural isomorphism defines an equivalence
relation on the set of functors $\integers \to G$, and, therefore, an
equivalence relation on $G$ itself.  What is this equivalence relation, in
purely group-theoretic terms?

(First have a guess.  For a general group $G$, what equivalence
relations on $G$ can you think of?)
\end{question}


\begin{question}        
\label{ex:species}
A \demph{permutation}%
%
\index{permutation}
%
of a set $X$ is a bijection $X \to X$.  Write $\Sym(X)$ for the set of
permutations of $X$.  A \demph{total%
%
\index{total order}%
\index{ordered set!totally}
%
order} on a set $X$ is an order $\leq$ such that for all $x, y \in X$,
either $x \leq y$ or $y \leq x$; so a total order on a finite set amounts
to a way of placing its elements in sequence.  Write $\Ord(X)$ for the set
of total orders on $X$.

Let $\cat{B}$ denote the category of finite sets and bijections.

\begin{enumerate}[(b)]
\item 
Give a definition of $\Sym$ on maps in $\cat{B}$ in such a way that $\Sym$
becomes a functor $\cat{B} \to \Set$.  Do the same for $\Ord$.  Both your
definitions should be canonical (no arbitrary choices).

\item 
Show that there is no natural transformation $\Sym \to \Ord$.
(Hint: consider identity permutations.)

\item 
For an $n$-element set $X$, how many elements do the sets $\Sym(X)$
and\linebreak $\Ord(X)$ have?
\end{enumerate}

Conclude that $\Sym(X) \iso \Ord(X)$ for all $X \in \cat{B}$, but not
\emph{naturally} in $X \in \cat{B}$.  (The moral is that for each finite
set $X$, there are exactly as many permutations of $X$ as there are total
orders on $X$, but there is no natural way of matching them up.)
\end{question} 


\begin{question} 
\label{ex:eqv-ffeso}
In this exercise, you will prove Proposition~\ref{propn:eqv-ffeso}.  Let $F
\from \cat{A} \to \cat{B}$ be a functor.
% 
\begin{enumerate}[(b)]
\item 
Suppose that $F$ is an equivalence.  Prove that $F$ is full, faithful and
essentially surjective on objects.  (Hint: prove faithfulness before
fullness.)

\item 
Now suppose instead that $F$ is full, faithful and essentially surjective
on objects.  For each $B \in \cat{B}$, choose an object $G(B)$ of $\cat{A}$
and an isomorphism $\epsln_B\from F(G(B)) \to B$.  Prove that $G$ extends
to a functor in such a way that $(\epsln_B)_{B \in \cat{B}}$ is a natural
isomorphism $FG \to 1_{\cat{B}}$.  Then construct a natural isomorphism
$1_{\cat{A}} \to GF$, thus proving that $F$ is an equivalence.
\end{enumerate}
\end{question}


\begin{question}
This exercise makes precise the idea that linear algebra can equivalently
be done with matrices%
%
\index{matrix}
%
or with linear maps.

Fix a field $k$.  Let $\Mt$ be the category whose objects are the natural
numbers and with
\[
\Mt(m, n) 
=
\{ n \times m \text{ matrices over } k \}.
\]
Prove that $\Mt$ is equivalent to $\FDVect$, the category of
finite-dimensional vector%
%
\index{vector space}
%
spaces over $k$.  Does your equivalence involve a \emph{canonical} functor
from $\Mt$ to $\FDVect$, or from $\FDVect$ to $\Mt$?

(Part of the exercise is to work out what composition in the category $\Mt$
is supposed to be; there is only one sensible possibility.
Proposition~\ref{propn:eqv-ffeso} makes the exercise easier.)
\end{question}


\begin{question}        
\label{ex:eqv-eq-reln}
Show that equivalence of categories is an equivalence relation.  (Not as
obvious as it looks.)
\end{question}
